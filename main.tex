%!TEX TS-program = xelatex
%---------------------------------------------------------------------------%
%->> 深圳大学毕业论文模板
%---------------------------------------------------------------------------%
%- 载入模板类
\documentclass{szuthesis}% 默认形式
% \documentclass[print]{szuthesis}% 打印预览版本,可以自动生成额外的空白页用于打印
% \documentclass[fontset=windows|adobe|mac|ubuntu]{szuthesis}% 选择字库
%---------------------------------------------------------------------------%
% - 载入配置信息,包含论文封面信息、必要的package
%---------------------------------------------------------------------------%
%->> Cover information 封面信息
%---------------------------------------------------------------------------%
\classid{\ TP39\;}% 分类号
\udc{\ \ \ 004\ \ }
\confidential{\ \ \ \ \ 公开}% 密级
%- 注:\title包含两个参数
% \title{深圳大学\LaTeX{}模板}{}% 单行题目,第二个参数为空
\title{aaaaaaaaaaaaaaaaaaaaaa}{bbbbbbbbbbbbbb}% 多行题目
%- 注:英文题目用于生成Abstract的页眉,只有一个参数
\TITLE{xxxxxxxxxxxxxxxxxxxxxxxxxxxxxxxxxxxx}
\author{\ \ \ \ \ \ \ \ xxx}% 论文作者
\idnumber{\ \ \ \ \ \ \ \ xxxxxxxxxx}
\major{\ \ \ \ \ \ \ \ xxxxxxx}% 学科专业名称
\dtype{\ \ \ \ \ \ \ \ 工学}% 学科门类名称
%- 注:以下两个类型支持多行
\institute{\ \ \ \ \ \ \ \ 计算机与软件学院}% 院系名称单行
% \institute{某某学院\\某某实验室}% 院系名称多行
\advisor{\ \ \ \ \ \ \ \ xxx}% 指导教师单行
% \advisor{张老师\ 教授\\王老师\ 研究员}% 指导教师多行
% !!! 记得切换学硕专硕
% \DEGREE{MasterXS}% 学术硕士
\DEGREE{MasterZY}% 专业硕士
%---------------------------------------------------------------------------%
%->> other config
%---------------------------------------------------------------------------%
%- 添加两个命令方便输出
\DeclareRobustCommand\cs[1]{\texttt{\char`\\#1}}
\providecommand\pkg[1]{{\sffamily#1}}
%-
\addbibresource{Biblio/ref.bib}% 参考文献路径
\setlength\bibitemsep{0.0ex plus 0.2ex minus 0.2ex}% set distance between bib entrie
%-
\setcounter{tocdepth}{3}% depth for the table of contents,设为2可不显示subsubsection
\setcounter{secnumdepth}{3}% depth for section numbering, default is 2
%-
%- 某些小语种会超出版面边界,提示Overfull \hbox{}...,中英日韩无需使用(或使用宏包microtype)
% \setlength\emergencystretch{1em}
%-
%- 重新设置 equation, figure, table 的序号
%\numberwithin{equation}{section}% set enumeration level
%\renewcommand{\theequation}{\thesection\arabic{equation}}% configure the label style
%\numberwithin{figure}{section}% set enumeration level
%\renewcommand{\thefigure}{\thesection\arabic{figure}}% configure the label style
%\numberwithin{table}{section}% set enumeration level
%\renewcommand{\thetable}{\thesection\arabic{table}}% configure the label style
\counterwithout{footnote}{chapter}% footnote编号全局连续
%-
%---------------------------------------------------------------------------%
%->> Package
%---------------------------------------------------------------------------%
% -> szuthesis.cls中已经导入的包
% - etoolbox, a toolbox of programming facilities
% - geometry, for layout
% - expl3, LaTeX3 programming environment
% - array
% - ulem, underline
% - xeCJKfntef, underline for CJK
% - fancyhdr, header and footer
% - biblatex
%-


\usepackage{graphicx}
\DeclareGraphicsExtensions{.pdf,.jpg,.png,.eps,.tif,.bmp}% 默认图片格式
\graphicspath{{Image/}}% 默认图片检索路径
%-
\usepackage[format=plain,hangindent=2.0em,font={small},skip=8pt,labelsep=space]{caption}
%-
\usepackage{subcaption}% 处理子图
%-
% \usepackage[list=off]{bicaption} % 双语caption
% \DeclareCaptionOption{bi-second}[]{
%     \def\tablename{Table}%
%     \def\figurename{Figure}%
% }
% \captionsetup[bi-second]{bi-second}
%-
\usepackage[section]{placeins}% 阻止图片浮动超出当前section
%-
\usepackage{enumitem}% 列表环境功能提升
\setlist{nosep}% 默认文本行间距
% \setlist[enumerate]{wide=\parindent}% 是否悬挂对齐,不建议全局修改
% \setlist[itemize]{wide=\parindent}
%-
% \usepackage{verbatim}
%-
% \usepackage{chemfig}% draw 2D chemical structures
% \usepackage[version=4]{mhchem}% typeset chemical formulae [mhchem|chemformula]
%-
% \usepackage{microtype}% improves general appearance of the text, 启用后降低编译效率
%-
% \usepackage{pdflscape}% landscape environment, \begin{landscape} ... \end{landscape}
%-
% \usepackage[usenames,dvipsnames,svgnames,table]{xcolor}% color support
%-
% \usepackage{tikz}% automatically load pgf package, plot with tex
% \usetikzlibrary{positioning, arrows, calc, trees }%
%-
\usepackage{booktabs}% 三线表
%-
\usepackage{listings}% 代码片段
\def\lstlistingname{代码}
\lstset{%
    basicstyle=\linespread{1.2}\small, % 字体
    breaklines=true,                   % 自动换行
    frame=lines,                       % 上下的边框,可选none|single|shadowbox等
    keepspaces=true,
    showstringspaces=false,            % string的空格添加标记,defaul:true
    tabsize=2,                         % tab长度
    % stringstyle=\color{DarkViolet},
    % backgroundcolor=\color{gray!10},
    % commentstyle=\color{ForestGreen},
    % keywordstyle=\color{blue},
}
%-
%%%%%%%%%%%%%%%%%%%%%%%%%%%%%%%%%%%%%%%%%%%%%%%%%%%%%%%%%%%%%%%%%%%%%%%%%%%%%%%%%%%%%%%%%%%%%%%%%%%%%%%%%%%%%%%%
% 自用package建议注释掉没有用的包,有些包会产生冲突,但不建议全部注释,有一些可能是通用的。如果不知道要不要注释可以全部注释重新添加需要的包,或者全部保留编译有问题再排除。
% \usepackage{multirow}
% \usepackage{hyperref}
% \usepackage{makecell}
% \usepackage{adjustbox}
% \usepackage{amsmath,amssymb}
% \usepackage{array}
% \usepackage{booktabs}
% % \usepackage{longtable},amsfonts
% \usepackage{tabularx}
% \usepackage{algorithmic}
% \usepackage{diagbox}
%%%%%%%%%%%%%%%%%%%%%%%%%%%%%%%%%%%%%%%%%%%%%%%%%%%%%%%%%%%%%%%%%%%%%%%%%%%%%%%%%%%%%%%%%%%%%%%%%%%%%%%%%%%%%%%%

\usepackage[ruled,vlined,linesnumbered]{algorithm2e} % 算法描述
\SetAlgorithmName{算法}{算法}{}
\SetArgSty{textit}
%---------------------------------------------------------------------------%
%->> 配置数学环境
%---------------------------------------------------------------------------%
\usepackage{amsmath,amssymb}
% \usepackage{pifont}
\newcommand*{\dif}{\mathop{}\!\mathrm{d}}
%- 符号表,参考 http://milde.users.sourceforge.net/LUCR/Math/mathpackages/amssymb-symbols.pdf
\usepackage{amsthm} % 定理引理等环境
\theoremstyle{plain}% for theorems, lemmas, propositions, etc
\newtheorem{theorem}              {定理} [chapter]
\newtheorem{axiom}      [theorem] {公理}
\newtheorem{lemma}      [theorem] {引理}
\newtheorem{corollary}  [theorem] {推论}
\newtheorem{assertion}  [theorem] {断言}
\newtheorem{proposition}[theorem] {命题}
\newtheorem{conjecture} [theorem] {猜想}
\newtheorem{assumption} [theorem] {假设}
\theoremstyle{definition}% for definitions and examples
\newtheorem{definition}           {定义} [chapter]
\newtheorem{example}              {例}   [chapter]
\newtheorem{problem}              {问题} [chapter]
\newtheorem{exercise}             {练习} [chapter]
\theoremstyle{remark}% for remarks and notes
\newtheorem*{remark}              {注}
\newtheorem*{solution}            {解}
% \usepackage{mathtools}
\usepackage{unicode-math}
%- 注:unicode-math可以配置数学公式字体,注意包冲突!
%- 已知可能存在冲突的包:amscd,amsfonts,bbm,bm,eucal,eufrak,mathrsfs
\setmathfont{XITSMath-Regular}[
    Extension=.otf, BoldFont=XITSMath-Bold, Ligatures=TeX, StylisticSet = 1,
]
\setmathfont{XITSMath-Regular}[
    Extension=.otf, range={scr,bfscr}, Ligatures=TeX, StylisticSet = 2,
]
\setmathfont{XITSMath-Regular}[
    Extension=.otf, range={cal,bfcal}, Ligatures=TeX, StylisticSet = 1,
]
% \setmathfont{XITS Math Bold}[version=bold]% for bold version % 不兼容StylisticSet=2
% \newenvironment{szumathbf}{\bfseries\mathversion{bold}}{}
\def\XITSMathFontOptions{
    Extension=.otf, BoldFont=XITSMath-Bold, Ligatures=TeX, StylisticSet = 1
}
\setmathrm{XITSMath-Regular}[\XITSMathFontOptions]
\setmathsf{XITSMath-Regular}[\XITSMathFontOptions]
\setmathtt{XITSMath-Regular}[\XITSMathFontOptions]
%-
\def\boldsymbol#1{\symbfit{#1}}
\providecommand{\Vector}[1]{\symbfit{#1}}
\providecommand{\Matrix}[1]{\symbfit{#1}}
\providecommand{\Tensor}[1]{\symbfit{#1}}
\providecommand{\Dif}{\symrm{d}}
\providecommand{\Const}[1]{\symrm{#1}}
\providecommand{\deltarm}{\symrm{\delta}}
\providecommand{\Div}{\operatorname{div}}
\providecommand{\Trace}{\operatorname{tr}}
%---------------------------------------------------------------------------%
%->> 链接,生成书签,在最后
%---------------------------------------------------------------------------%
\usepackage{hyperref}% 超链接,生成书签,[注:放在最后]
\hypersetup{% set hyperlinks
    pdfencoding=auto,% allows non-Latin based languages in bookmarks
    psdextra=true,% extra support for math symbols in bookmarks
    bookmarksnumbered=true,% put section numbers in bookmarks
    pdftitle={\szutitle},% title
    pdfauthor={\szuauthor},% author
    pdfsubject={\szutitle},% subject
    pdfstartview={FitH},% fits the width of the page to the window
    % colorlinks=true,% false: boxed links; true: colored links
    % linkcolor=black,% color of internal links
    % citecolor=blue,% color of links to bibliography
    % filecolor=blue,% color of file links
    % urlcolor=blue,% color of external links
    hidelinks,% hide links color and box
}
%---------------------------------------------------------------------------%
%->> END
%---------------------------------------------------------------------------%
%---------------------------------------------------------------------------%
%- 辅助命令,后文中的所有\include均可在此单独列出,用逗号隔开,
%- 以此只编译必要的章节,加快编译速度,待全文完成后可注释本命令,即可编译全文。
%- 也可注释部分内容,正文中所有的内容均可注释后避免其参与编译,包含maketitle等命令,
%- 执行此命令或注释后可能导致章节序号发生错误,无需担心,全文编译后即可恢复
% \includeonly{Tex/Abstract,Tex/Appendix}
% %---------------------------------------------------------------------------%
\begin{document}
%-
\maketitle% 制作封面
%-
%- 声明包含两种形式,
%- 如果参数为空则可自动生成默认声明页,也可设置参数导入签字后的扫描版PDF文件
% \makedeclaration{declaration}% 制作声明,参数为扫描版文件名,默认在Image下
\makedeclaration{}% 制作声明,自动生成
%-
\frontmatter% 初始化摘要页环境,不建议注释
%-
%---------------------------------------------------------------------------%
%->> Abstract
%---------------------------------------------------------------------------%
%-
%-> 中文摘要
%-
\begin{abstract}
    \markboth{摘\ \ 要}{}
中文摘要
    
\keywords{关键词一,关键词二}% 中文关键词
\end{abstract}
%-
%-> 英文摘要
%-
\begin{ABSTRACT}
ABSTRACT


\KEYWORDS{keyword1, keyword2}% 英文关键词
\end{ABSTRACT}
%---------------------------------------------------------------------------%
%-
\tableofcontents% 目录
%-
\mainmatter% 初始化正文环境,不建议注释
%-
\chapter{绪论}\label{chap:intro}
\markboth{第一章\ \ 绪论}{}
\section{研究背景}\label{sec:background}

当前模板完美遵循《学术学位硕士学位论文印刷格式样式》与《专业硕士学位论文印刷格式样式》中规定的学位论文撰写要求和封面设定。
目前支持 Windows 操作系统(Linux、MacOS可能会有未知问题);目前仅支持 Xe\LaTeX{} 引擎;文献编译引擎biber (biblatex)。
支持中文书签、中文渲染、拷贝 PDF 中的文本到其他文本编辑器等特性。


\section{系统要求}\label{sec:system}

szuthesis 宏包可以在目前主流的 \href{https://en.wikibooks.org/wiki/LaTeX/Introduction}{\LaTeX{}} 发行版中使用,
如 \TeX{}Live 和 MiK\TeX{}。因 C\TeX{} 套装已停止维护,\textbf{请勿使用}。
请勿混淆 C\TeX{} 套装\footnote{\url{http://www.ctex.org/CTeX}}与 C\TeX{} 宏集\footnote{\url{https://ctan.org/pkg/ctex?lang=en}}。
C\TeX{} 套装基于 Windows 下的 MiKTeX 开发,在其基础上增加了对中文的完整支持,已于 2012 年起停止维护。
而 C\TeX{} 宏集是通用 \LaTeX{} 排版框架,为中文 \LaTeX{} 文档提供了汉字支持,主要包括宏包 ctex 以及中文文档类 ctexart、 ctexbook 等。

推荐的 \LaTeX{} 发行版如下:

\begin{center}
    %\footnotesize% fontsize
    %\setlength{\tabcolsep}{4pt}% column separation
    %\renewcommand{\arraystretch}{1.5}% row space 
    \begin{tabular}{lc}
        \toprule
        操作系统         & \LaTeX{}发行版                                                                                        \\
        \midrule
        Linux 或 Windows & \href{https://www.tug.org/texlive/}{\TeX{}Live Full} 或 \href{https://miktex.org/download}{MiK\TeX{}} \\
        MacOS            & \href{https://www.tug.org/mactex/}{Mac\TeX{} Full} 或 \href{https://miktex.org/download}{MiK\TeX{}}   \\
        \bottomrule
    \end{tabular}
\end{center}

请从各软件官网下载安装程序,勿使用不明程序源。若系统原带有旧版的 \LaTeX{} 发行版并想安装新版,请\textbf{先完全卸载旧版再安装新版}。
推荐安装2019年后的版本。可能因为网络问题导致安装速度较慢,推荐在安装时无需选择额外宏包,
安装完成后添加清华源\footnote{\url{https://mirrors.tuna.tsinghua.edu.cn/help/CTAN/}},再继续安装所需宏包。

如选择部分安装,请安装后检测以下宏包知否安装,若未安装导致的BUG不易排查:

\begin{center}
    \small% fontsize
    \renewcommand{\arraystretch}{0.8}% row space 
    \begin{tabular}{ll}
        \toprule
        宏包                 & 功能                    \\
        \midrule
        xits                 & 开源Times New Roman字体 \\
        biber                & biblatex引擎            \\
        biblatex-gb7714-2015 & biblatex格式            \\
        latexmk              & 自动编译latex文档       \\
        \bottomrule
    \end{tabular}
\end{center}

安装 \LaTeX{} 发行版后,即可使用任意编辑器开始书写。
在这里推荐使用 VS Code\footnote{\url{https://code.visualstudio.com/}} 作为编辑器。一方面其可单纯的作为编辑器使用,
同时又可以搭配插件进行扩展。可以搭配 Git 进行版本控制,又可以安装 LaTeX Workshop 插件直接进行编译。
LaTeX Workshop 插件提供了诸如 Linting,Formatting,Intellisense,PDF 文件预览,公式预览,
全文大纲等诸多功能\footnote{\url{https://github.com/James-Yu/LaTeX-Workshop\#features-taster}}。
使用快捷键可以极大提高编写效率,例如使用 \lstinline!Ctrl+Alt+j! 可以快速从 tex 文本跳转到 PDF 中对应的位置,
而在PDF预览中使用 \lstinline!Ctrl+鼠标左键! 就可以快速定位对应的 tex 文本。


\section{编译}

\begin{enumerate}[wide=\parindent]
    \item 安装软件:根据所用操作系统和章节~\ref{sec:system} 中的信息安装 \LaTeX{} 编译环境。

    \item 获取模板:下载 szuthesis 项目。szuthesis 不仅提供了相应的模板,同时也提供了编译样例,
          下载时推荐下载整个 szuthesis 文件夹,而不是单独的 cls 文档类。

    \item 编译模板:参考项目主页编译部分。
\end{enumerate}

编译完成后即可获得这份说明文档。而这也完成了学习使用 szuthesis 撰写论文的一半进程。


\section{文档目录简介}

\subsection{Thesis.tex}

Thesis.tex 为主文档,包含了论文全篇的主要架构。其中,document 中的所有内容均可注释后避免其参与编译,包含 maketitle 等命令,
注释后可能导致章节序号发生错误,无需担心,全文编译后即可恢复。注释后可加快编译速度,例如参考文献页无须随文档实时编译,
只需要全文完成后编译参考文献页即可,这也是使用 \LaTeX{} 编写文档的优点之一。

\subsection{Temp文件夹}

编译后,生成的临时文件皆存于Temp文件夹内,包括编译得到的 PDF 文档,其存在是为了保持工作空间的整洁,因为好的心情是很重要的。

\subsection{szuthesis.cls}

\verb!texmf\tex\latex\szuthesis\szuthesis.cls! 目录下的 szuthesis.cls 为文档类,定义了论文的核心格式,
包括论文排版、引用格式、页眉页脚、字体字号等。其中,根据《印刷格式样式》规定,参考文献后的字号均与正文字号不同。

\subsection{config.tex}

szuthesis.cls 需要传入一些参数用来生成封面信息,config.tex 可用来传入这些参数。
后边则定义了一些可选的宏包,这些宏包并不完全属于《印刷格式样式》规定的排版,可以自由选择是否启用。
例如数学公式的字体、代码片段、超链接等,均在 config.tex 进行了定义,这些可以根据需要对它进行调整。

\subsection{Tex文件夹}

文件夹内为论文的所有正文内容,这也是使用 szuthesis 撰写学位论文时,主要关注和修改的一个位置。
\textbf{注:所有文件都必须采用 UTF-8 编码,否则编译后将出现乱码文本},详细分类介绍如下:

\begin{itemize}
    \item Abstract.tex:摘要信息。
    \item ChapterX.tex:论文的各个章节,可根据需要添加和撰写。\textbf{添加新章节时,注意编码格式,可拷贝一个已有的章文件再重命名,以继承文档的 UTF-8 编码}。
    \item Appendix.tex:附录,注意附录字号与正文不同,仅用于添加补充信息,如有整段文本建议放置于正文中。
    \item Acknowledgements.tex:致谢。
    \item Publications.tex: 研究成果。
\end{itemize}

\subsection{Image文件夹}

用于放置论文中所需要的图形类文件,支持格式有:jpg, png, pdf 等,需要更多支持格式可在 config.tex 中配置。
不建议为各章节图片建子目录,即使图片众多,若命名规则合理,图片查询亦是十分方便。

\subsection{Biblio文件夹}

ref.bib 为参考文献信息,可在 config.tex 中配置。

\subsection{.vscode文件夹}

这一文件夹用于保存 VS Code 的配置文件,其中 settings.json 保存了部分 latexmk 所需的配置项。



\section{帮助与问题反馈}\label{sec:help}

对于 \LaTeX{} 相关问题,推荐使用 texdoc 命令查阅相关文档。
例如安装 lshort-chinese 宏包后,可使用 \textbf{texdoc lshort-chinese} 命令打开一份教程,包含了 \LaTeX{} 入门相关的知识。
使用 \textbf{texdoc ctex} 则可打开 ctex 宏包的文档,包含中文排版相关的内容,例如第5节中则详细介绍了中文字体字号。
大多数宏包都提供了非常详尽的文档,都可以使用 texdoc 查阅。

欢迎各位同学提出宝贵意见,一起不断改进模板。
% \include{Tex/Chapter2}
%- 
\backmatter% 初始化其他部分环境,不建议注释
%-
\szubibliography% 导入参考文献
%-
\chapter[附录]{附录A\ \ xxxxxx}
\markboth{附录A\ \ xxxxxx}{}

% 导入附录
\chapter*{附录B\ \ xxxxxx}
\markboth{附录B\ \ xxxxxx}{}

%-
%- 2020年新增,添加答辩记录,建议分成三个独立的PDF文件,
%- \szuaddpdf命令包含两个参数,[]中为可选参数,用于生成目录,{}中为PDF文件名,默认在Image下
% \szuaddpdf[指导教师对研究生学位论文的学术评语]{pingyu}
% \szuaddpdf[学位论文答辩委员会决议书]{dabian1}
%-\szuaddpdf{dabian2}% 前一页生成目录即可
%-
\chapter[致谢]{致\ \ 谢}
\markboth{致谢}{}
% 导入致谢
%-
\chapter{攻读硕士学位期间的科研成果}
\markboth{攻读硕士学位期间的科研成果}{}


%- 可以直接使用引用的格式

\begin{enumerate}[label = {[\arabic*]}]
    \item[] {\zihao{-3}\heiti 学术论文}
    \zihao{5}
    \item \underline{Chen, J.}, . Encouraging Sparsity xxxxxxxxx
\end{enumerate}
% 导入研究成果
%-
\end{document}
% %---------------------------------------------------------------------------%
